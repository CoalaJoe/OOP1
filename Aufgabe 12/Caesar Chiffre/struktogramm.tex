\documentclass{article}

\usepackage{struktex}

\begin{document}

\begin{struktogramm}(120,70)[Decrypt(char Array, Schl\"ussel)]
    \while{L\"ange des char[]}
        \ifthenelse{5}{2}{Index des char[] ist Null-Terminator}{y}{n}
            \assign{Schleife stoppen}
        \change
            \assign{Char in char[] als int casten.}
            \assign{int plus schl\"ussel.}
            \assign{int als char casten und in char[] einf\"ugen.}
        \ifend
    \whileend
    \exit{R\"uckgabe des char[].}
\end{struktogramm}

\begin{struktogramm}(120,70)[Encrypt(char Array, Schl\"ussel)]
    \while{L\"ange des char[]}
        \ifthenelse{5}{2}{Index des char[] ist Null-Terminator}{y}{n}
            \assign{Schleife stoppen}
        \change
            \assign{Char in char[] als int casten.}
            \assign{int minus schl\"ussel.}
            \assign{int als char casten und in char[] einf\"ugen.}
        \ifend
    \whileend
    \exit{R\"uckgabe des char[].}
\end{struktogramm}

\begin{struktogramm}(120,40)[main]
    \assign{Ausgabe: "Schl\"ussel eingeben: "}
    \assign{Eingabe: 6}

    \assign{Ausgabe: "Wort eingeben: "}
    \assign{Eingabe: "programmieren"}

    \assign{Ausgabe: "Schl\"ussel lautet: "}
    \assign{Ausgabe: Schl\"usselvariable}

    \assign{Ausgabe: "Der verschl\"usselte Text lautet : "}
    \sub{encrypt}
    \assign{Ausgabe: verschl\"usselte Eingabe.}

    \sub{decrypt}
    \assign{Ausgabe: "Der entschl\"usselte Text lautet : "}
    \assign{Ausgabe: entschl\"usselte verschl\"usselung.}
\end{struktogramm}

\end{document}
